\documentclass{ctexart}

\usepackage[colorlinks, linkcolor = red]{hyperref}

\usepackage{tcolorbox}
\tcbuselibrary{skins}
\usepackage{colortbl}
\def\mydollar{{\fontspec{Consolas}\$\ }}% 单个字体设置
\definecolor{structure}{RGB}{0,120,2}
\definecolor{main}{RGB}{0,166,82}
\definecolor{second}{RGB}{230,90,7}
\definecolor{white}{RGB}{255,255,255}

\newtcolorbox{mybox}[2][]{
    colframe = main,  
    colback = main!5,
    colbacktitle = main,
    coltitle = white, 
    title={#2},
    fonttitle=\bfseries,
    #1
}

\title{Arch Linux 学习笔记}
\author{ZiTai}
\date{\zhdate{2023/7/9}}

\begin{document}
\maketitle

\clearpage
\tableofcontents

\clearpage
\section{起步}
    使用 man 帮助手册,可以查询命令的使用。
    
    例如,在 Linux 命令行里运行 \texttt{man man} 命令,可以查询命令“man”的用法。

\clearpage
\section{常用 Bash 命令}
\begin{mybox}{文件管理}
    \texttt{cat}

    \texttt{file}

    \texttt{find}

    \texttt{less}

    \texttt{locate}

    \texttt{more}

    \texttt{mv}

    \texttt{rm}

    \texttt{touch}

    \texttt{which}

    \texttt{cp}

    \texttt{whereis}

    \texttt{read}
\end{mybox}

\begin{mybox}{文档编辑}
    \texttt{fold}

    \texttt{grep}
\end{mybox}

\begin{mybox}{磁盘管理}
    \texttt{cd}

    \texttt{df}

    \texttt{mkdir}

    \texttt{pwd}

    \texttt{stat}

    \texttt{tree}

    \texttt{ls}
\end{mybox}

\begin{mybox}{系统管理}
    \texttt{date}

    \texttt{exit}

    \texttt{sleep}

    \texttt{halt}
    
    \texttt{kill}

    \texttt{login}

    \texttt{logname}

    \texttt{logout}

    \texttt{top}

    \texttt{reboot}

    \texttt{shutdown}
    
    \texttt{sudo}

    \texttt{chsh}

    \texttt{who}

    \texttt{whoami}

    \texttt{whois}
\end{mybox}

\begin{mybox}{系统设置}
    \texttt{clear}

    \texttt{alias}

    \texttt{unalias}

    \texttt{dircolors}

    \texttt{bind}

    \texttt{clock}

    \texttt{export}

    \texttt{passwd}

    \texttt{time}
\end{mybox}

\begin{mybox}{备份压缩}
    \texttt{gunzip}

    \texttt{dump}
    
    \texttt{gzip}

    \texttt{restore}

    \texttt{tar}
    
    \texttt{unzip}

    \texttt{zip}

    \texttt{zipinfo}
\end{mybox}


\begin{mybox}{设备管理}
    \texttt{poweroff}
\end{mybox}

\clearpage
\section{ArchLinux 常见操作}



% \begin{mybox}{\href{https://www.runoob.com/linux/linux-comm-cat.html}{cat}}
%     用于连接文件并打印到标准输出设备上
%     \tcbline
%     \texttt{cat [-AbeEnstTuv] [--help] [--version] filename}
%     \tcbline
%     参数说明:

%     \begin{itemize}
%       \item -n 或 --number:由 1 开始对所有输出的行数编号。
%       \item -b 或 --number-nonblank:和 -n 相似,只不过对于空白行不编号。
%       \item -s 或 --squeeze-blank:当遇到有连续两行以上的空白行,就代换为一行的空白行。
%       \item -v 或 --show-nonprinting:使用 \^ 和 M- 符号,除了 LFD 和 TAB 之外。
%       \item -E 或 --show-ends : 在每行结束处显示 \$。
%       \item -T 或 --show-tabs: 将 TAB 字符显示为 \^I。
%       \item -A, --show-all:等价于 -vET。
%       \item -e:等价于"-vE"选项;
%       \item -t:等价于"-vT"选项;
%     \end{itemize}
% \end{mybox}

% \begin{mybox}{\href{https://www.runoob.com/linux/linux-comm-chmod.html}{chmod}}
%     控制用户对文件的权限的命令
%     \tcbline
%     \texttt{chmod [-cfvR] [--help] [--version] mode file...}
% \end{mybox}

% \begin{mybox}{\href{https://www.runoob.com/linux/linux-comm-file.html}{file}}
%     用于辨识文件类型
%     \tcbline
%     \texttt{file [-bcLvz][-f <名称文件>][-m <魔法数字文件>...][文件或目录...]}
%     \tcbline
%     参数说明:

%     \begin{itemize}
%       \item -b 列出辨识结果时,不显示文件名称。
%       \item -c 详细显示指令执行过程,便于排错或分析程序执行的情形。
%       \item -f<名称文件> 指定名称文件,其内容有一个或多个文件名称时,让 file 依序辨识这些文件,格式为每列一个文件名称。
%       \item -L 直接显示符号连接所指向的文件的类别。
%       \item -m<魔法数字文件> 指定魔法数字文件。
%       \item -v 显示版本信息。
%       \item -z 尝试去解读压缩文件的内容。
%       \item [文件或目录...] 要确定类型的文件列表,多个文件之间使用空格分开,可以使用 shell 通配符匹配多个文件。
%     \end{itemize}
% \end{mybox}

% \begin{mybox}{\href{https://www.runoob.com/linux/linux-comm-find.html}{find}}
%     \tcbline
%     \texttt{find [path] [expression]}
%     \tcbline
%     参数说明:

%     path 是要查找的目录路径,可以是一个目录或文件名,也可以是多个路径,多个路径之间用空格分隔,如果未指定路径,则默认为当前目录。

%     expression 是可选参数,用于指定查找的条件,可以是文件名、文件类型、文件大小等等。
% \end{mybox}

% \begin{mybox}{\href{https://www.runoob.com/linux/linux-comm-less.html}{less}}
%     less 与 more 类似,less 可以随意浏览文件,支持翻页和搜索,支持向上翻页和向下翻页
%     \tcbline
%     \texttt{less [参数] 文件 }
%     \tcbline
%     参数说明:

%     \begin{itemize}
%         \item -b <缓冲区大小> 设置缓冲区的大小
%         \item -e 当文件显示结束后,自动离开
%         \item -f 强迫打开特殊文件,例如外围设备代号、目录和二进制文件
%         \item -g 只标志最后搜索的关键词
%         \item -i 忽略搜索时的大小写
%         \item -m 显示类似more命令的百分比
%         \item -N 显示每行的行号
%         \item -o <文件名> 将less 输出的内容在指定文件中保存起来
%         \item -Q 不使用警告音
%         \item -s 显示连续空行为一行
%         \item -S 行过长时间将超出部分舍弃
%         \item -x <数字> 将"tab"键显示为规定的数字空格
%         \item /字符串:向下搜索"字符串"的功能
%         \item ?字符串:向上搜索"字符串"的功能
%         \item n:重复前一个搜索(与 / 或 ? 有关)
%         \item N:反向重复前一个搜索(与 / 或 ? 有关)
%         \item b 向上翻一页
%         \item d 向后翻半页
%         \item h 显示帮助界面
%         \item Q 退出less 命令
%         \item u 向前滚动半页
%         \item y 向前滚动一行
%         \item 空格键 滚动一页
%         \item 回车键 滚动一行
%         \item [pagedown]: 向下翻动一页
%         \item [pageup]: 向上翻动一页
%     \end{itemize}
% \end{mybox}
\end{document}
